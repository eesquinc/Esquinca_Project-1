\documentclass[]{article}
\usepackage{lmodern}
\usepackage{amssymb,amsmath}
\usepackage{ifxetex,ifluatex}
\usepackage{fixltx2e} % provides \textsubscript
\ifnum 0\ifxetex 1\fi\ifluatex 1\fi=0 % if pdftex
  \usepackage[T1]{fontenc}
  \usepackage[utf8]{inputenc}
\else % if luatex or xelatex
  \ifxetex
    \usepackage{mathspec}
  \else
    \usepackage{fontspec}
  \fi
  \defaultfontfeatures{Ligatures=TeX,Scale=MatchLowercase}
\fi
% use upquote if available, for straight quotes in verbatim environments
\IfFileExists{upquote.sty}{\usepackage{upquote}}{}
% use microtype if available
\IfFileExists{microtype.sty}{%
\usepackage{microtype}
\UseMicrotypeSet[protrusion]{basicmath} % disable protrusion for tt fonts
}{}
\usepackage[margin=1in]{geometry}
\usepackage{hyperref}
\hypersetup{unicode=true,
            pdftitle={IQ Analysis},
            pdfborder={0 0 0},
            breaklinks=true}
\urlstyle{same}  % don't use monospace font for urls
\usepackage{color}
\usepackage{fancyvrb}
\newcommand{\VerbBar}{|}
\newcommand{\VERB}{\Verb[commandchars=\\\{\}]}
\DefineVerbatimEnvironment{Highlighting}{Verbatim}{commandchars=\\\{\}}
% Add ',fontsize=\small' for more characters per line
\usepackage{framed}
\definecolor{shadecolor}{RGB}{248,248,248}
\newenvironment{Shaded}{\begin{snugshade}}{\end{snugshade}}
\newcommand{\KeywordTok}[1]{\textcolor[rgb]{0.13,0.29,0.53}{\textbf{#1}}}
\newcommand{\DataTypeTok}[1]{\textcolor[rgb]{0.13,0.29,0.53}{#1}}
\newcommand{\DecValTok}[1]{\textcolor[rgb]{0.00,0.00,0.81}{#1}}
\newcommand{\BaseNTok}[1]{\textcolor[rgb]{0.00,0.00,0.81}{#1}}
\newcommand{\FloatTok}[1]{\textcolor[rgb]{0.00,0.00,0.81}{#1}}
\newcommand{\ConstantTok}[1]{\textcolor[rgb]{0.00,0.00,0.00}{#1}}
\newcommand{\CharTok}[1]{\textcolor[rgb]{0.31,0.60,0.02}{#1}}
\newcommand{\SpecialCharTok}[1]{\textcolor[rgb]{0.00,0.00,0.00}{#1}}
\newcommand{\StringTok}[1]{\textcolor[rgb]{0.31,0.60,0.02}{#1}}
\newcommand{\VerbatimStringTok}[1]{\textcolor[rgb]{0.31,0.60,0.02}{#1}}
\newcommand{\SpecialStringTok}[1]{\textcolor[rgb]{0.31,0.60,0.02}{#1}}
\newcommand{\ImportTok}[1]{#1}
\newcommand{\CommentTok}[1]{\textcolor[rgb]{0.56,0.35,0.01}{\textit{#1}}}
\newcommand{\DocumentationTok}[1]{\textcolor[rgb]{0.56,0.35,0.01}{\textbf{\textit{#1}}}}
\newcommand{\AnnotationTok}[1]{\textcolor[rgb]{0.56,0.35,0.01}{\textbf{\textit{#1}}}}
\newcommand{\CommentVarTok}[1]{\textcolor[rgb]{0.56,0.35,0.01}{\textbf{\textit{#1}}}}
\newcommand{\OtherTok}[1]{\textcolor[rgb]{0.56,0.35,0.01}{#1}}
\newcommand{\FunctionTok}[1]{\textcolor[rgb]{0.00,0.00,0.00}{#1}}
\newcommand{\VariableTok}[1]{\textcolor[rgb]{0.00,0.00,0.00}{#1}}
\newcommand{\ControlFlowTok}[1]{\textcolor[rgb]{0.13,0.29,0.53}{\textbf{#1}}}
\newcommand{\OperatorTok}[1]{\textcolor[rgb]{0.81,0.36,0.00}{\textbf{#1}}}
\newcommand{\BuiltInTok}[1]{#1}
\newcommand{\ExtensionTok}[1]{#1}
\newcommand{\PreprocessorTok}[1]{\textcolor[rgb]{0.56,0.35,0.01}{\textit{#1}}}
\newcommand{\AttributeTok}[1]{\textcolor[rgb]{0.77,0.63,0.00}{#1}}
\newcommand{\RegionMarkerTok}[1]{#1}
\newcommand{\InformationTok}[1]{\textcolor[rgb]{0.56,0.35,0.01}{\textbf{\textit{#1}}}}
\newcommand{\WarningTok}[1]{\textcolor[rgb]{0.56,0.35,0.01}{\textbf{\textit{#1}}}}
\newcommand{\AlertTok}[1]{\textcolor[rgb]{0.94,0.16,0.16}{#1}}
\newcommand{\ErrorTok}[1]{\textcolor[rgb]{0.64,0.00,0.00}{\textbf{#1}}}
\newcommand{\NormalTok}[1]{#1}
\usepackage{longtable,booktabs}
\usepackage{graphicx,grffile}
\makeatletter
\def\maxwidth{\ifdim\Gin@nat@width>\linewidth\linewidth\else\Gin@nat@width\fi}
\def\maxheight{\ifdim\Gin@nat@height>\textheight\textheight\else\Gin@nat@height\fi}
\makeatother
% Scale images if necessary, so that they will not overflow the page
% margins by default, and it is still possible to overwrite the defaults
% using explicit options in \includegraphics[width, height, ...]{}
\setkeys{Gin}{width=\maxwidth,height=\maxheight,keepaspectratio}
\IfFileExists{parskip.sty}{%
\usepackage{parskip}
}{% else
\setlength{\parindent}{0pt}
\setlength{\parskip}{6pt plus 2pt minus 1pt}
}
\setlength{\emergencystretch}{3em}  % prevent overfull lines
\providecommand{\tightlist}{%
  \setlength{\itemsep}{0pt}\setlength{\parskip}{0pt}}
\setcounter{secnumdepth}{0}
% Redefines (sub)paragraphs to behave more like sections
\ifx\paragraph\undefined\else
\let\oldparagraph\paragraph
\renewcommand{\paragraph}[1]{\oldparagraph{#1}\mbox{}}
\fi
\ifx\subparagraph\undefined\else
\let\oldsubparagraph\subparagraph
\renewcommand{\subparagraph}[1]{\oldsubparagraph{#1}\mbox{}}
\fi

%%% Use protect on footnotes to avoid problems with footnotes in titles
\let\rmarkdownfootnote\footnote%
\def\footnote{\protect\rmarkdownfootnote}

%%% Change title format to be more compact
\usepackage{titling}

% Create subtitle command for use in maketitle
\newcommand{\subtitle}[1]{
  \posttitle{
    \begin{center}\large#1\end{center}
    }
}

\setlength{\droptitle}{-2em}

  \title{IQ Analysis}
    \pretitle{\vspace{\droptitle}\centering\huge}
  \posttitle{\par}
    \author{}
    \preauthor{}\postauthor{}
    \date{}
    \predate{}\postdate{}
  

\begin{document}
\maketitle

\subsection{Stat Consulting: Hw 2}\label{stat-consulting-hw-2}

Author: Erika Esquinca Class: Bios 6621 Dr.~Ryan Peterson Date:
9.15.2020

The purpose of this assignment is to be able to create a new project in
R using version control and a git repository. This makes me work
reproducible as any changes, previous code, and directories are all
linked to github to store on the cloud.

This report is on IQ levels based on lead exposure and must contain the
following.

A. A graph showing the IQ levels by location status

B. At least one nicely formatted table using the `kable' function

C. A couple sentences of text (not on the graph, but in the body of the
Rmarkdown document) describing the graph and the table

D. Inline calculations giving the values of the means, that will be
updated if the data are changed,

E. At least one R code chunk.

A. A graph showing the IQ levels by location status

\begin{Shaded}
\begin{Highlighting}[]
\NormalTok{iq <-}\StringTok{ }\KeywordTok{read.csv}\NormalTok{(}\StringTok{"lead-iq-01.csv"}\NormalTok{)}
\end{Highlighting}
\end{Shaded}

Lets begin the analyses by looking at the first 6 rows of our data and
the structure

\begin{Shaded}
\begin{Highlighting}[]
\KeywordTok{str}\NormalTok{(iq)}
\end{Highlighting}
\end{Shaded}

\begin{verbatim}
## 'data.frame':    124 obs. of  2 variables:
##  $ Smelter: Factor w/ 2 levels "Far","Near": 1 1 1 1 1 1 1 1 1 1 ...
##  $ IQ     : int  70 85 86 76 96 94 115 97 128 999 ...
\end{verbatim}

\begin{Shaded}
\begin{Highlighting}[]
\KeywordTok{head}\NormalTok{(iq)}
\end{Highlighting}
\end{Shaded}

\begin{verbatim}
##   Smelter IQ
## 1     Far 70
## 2     Far 85
## 3     Far 86
## 4     Far 76
## 5     Far 96
## 6     Far 94
\end{verbatim}

Looks like Smelter is a factor with two level ``Far'' = 1 \& ``Near'' =
0 and IQ is an integer value.

\begin{Shaded}
\begin{Highlighting}[]
\KeywordTok{library}\NormalTok{(ggplot2)}
\end{Highlighting}
\end{Shaded}

\begin{verbatim}
## Warning: As of rlang 0.4.0, dplyr must be at least version 0.8.0.
## * dplyr 0.7.8 is too old for rlang 0.4.5.
## * Please update dplyr to the latest version.
## * Updating packages on Windows requires precautions:
##   <https://github.com/jennybc/what-they-forgot/issues/62>
\end{verbatim}

\begin{Shaded}
\begin{Highlighting}[]
\KeywordTok{qplot}\NormalTok{(}\DataTypeTok{x=}\NormalTok{ iq}\OperatorTok{$}\NormalTok{Smelter, }\DataTypeTok{y=}\NormalTok{ iq}\OperatorTok{$}\NormalTok{IQ, }
      \DataTypeTok{xlab=} \StringTok{"Smelter"}\NormalTok{,}
      \DataTypeTok{ylab=} \StringTok{"IQ"}\NormalTok{,}
      \DataTypeTok{ylim=} \KeywordTok{c}\NormalTok{(}\DecValTok{0}\NormalTok{,}\DecValTok{1000}\NormalTok{),}
      \DataTypeTok{main=} \StringTok{"Ranges of IQ when far or near smelter"}\NormalTok{)}
\end{Highlighting}
\end{Shaded}

\includegraphics{IQ_v_Lead_files/figure-latex/unnamed-chunk-3-1.pdf}

We can also interpret this data using boxplots, however notice that our
outputs are hard to read because of the outlier changing the scale

\begin{Shaded}
\begin{Highlighting}[]
\KeywordTok{plot}\NormalTok{(}\DataTypeTok{x=}\NormalTok{ iq}\OperatorTok{$}\NormalTok{Smelter, }\DataTypeTok{y=}\NormalTok{ iq}\OperatorTok{$}\NormalTok{IQ, }
      \DataTypeTok{xlab=} \StringTok{"Smelter"}\NormalTok{,}
      \DataTypeTok{ylab=} \StringTok{"IQ"}\NormalTok{,}
      \DataTypeTok{ylim=} \KeywordTok{c}\NormalTok{(}\DecValTok{0}\NormalTok{,}\DecValTok{1000}\NormalTok{),}
      \DataTypeTok{main=} \StringTok{"Ranges of IQ when far or near smelter"}\NormalTok{)}
\end{Highlighting}
\end{Shaded}

\includegraphics{IQ_v_Lead_files/figure-latex/unnamed-chunk-4-1.pdf}

B. At least one nicely formatted table using the `kable' function

\begin{Shaded}
\begin{Highlighting}[]
\KeywordTok{library}\NormalTok{(knitr)}
\KeywordTok{kable}\NormalTok{(}\KeywordTok{summary}\NormalTok{(iq))}
\end{Highlighting}
\end{Shaded}

\begin{longtable}[]{@{}llc@{}}
\toprule
& Smelter & IQ\tabularnewline
\midrule
\endhead
& Far :67 & Min. : 46.00\tabularnewline
& Near:57 & 1st Qu.: 81.50\tabularnewline
& NA & Median : 91.00\tabularnewline
& NA & Mean : 98.34\tabularnewline
& NA & 3rd Qu.: 99.25\tabularnewline
& NA & Max. :999.00\tabularnewline
\bottomrule
\end{longtable}

\begin{Shaded}
\begin{Highlighting}[]
\KeywordTok{kable}\NormalTok{(}\KeywordTok{summary}\NormalTok{(iq[iq}\OperatorTok{$}\NormalTok{Smelter }\OperatorTok{==}\StringTok{ "Far"}\NormalTok{,]))}
\end{Highlighting}
\end{Shaded}

\begin{longtable}[]{@{}llc@{}}
\toprule
& Smelter & IQ\tabularnewline
\midrule
\endhead
& Far :67 & Min. : 46.0\tabularnewline
& Near: 0 & 1st Qu.: 84.0\tabularnewline
& NA & Median : 93.0\tabularnewline
& NA & Mean :106.1\tabularnewline
& NA & 3rd Qu.:101.0\tabularnewline
& NA & Max. :999.0\tabularnewline
\bottomrule
\end{longtable}

\begin{Shaded}
\begin{Highlighting}[]
\KeywordTok{kable}\NormalTok{(}\KeywordTok{summary}\NormalTok{(iq[iq}\OperatorTok{$}\NormalTok{Smelter }\OperatorTok{==}\StringTok{ "Near"}\NormalTok{,]))}
\end{Highlighting}
\end{Shaded}

\begin{longtable}[]{@{}llc@{}}
\toprule
& Smelter & IQ\tabularnewline
\midrule
\endhead
& Far : 0 & Min. : 56.00\tabularnewline
& Near:57 & 1st Qu.: 80.00\tabularnewline
& NA & Median : 88.00\tabularnewline
& NA & Mean : 89.19\tabularnewline
& NA & 3rd Qu.: 96.00\tabularnewline
& NA & Max. :115.00\tabularnewline
\bottomrule
\end{longtable}

C. A couple sentences of text (not on the graph, but in the body of the
Rmarkdown document) describing the graph and the table

Notice now that the graphs ahow the IQ results obtained from individuals
that are either considered near smelter which is a subject being within
1 mile of lead emitting ore smelter or more than 1 mile considered far.
The most notable observation is that in the far category we have a
subject with an IQ of 999. This then means we have a larger range of
possible IQ's for the far category.

D. Inline calculations giving the values of the means, that will be
updated if the data are changed,

We can also now input values we can see using inline R code that get
updated when the document gets knitted. This way if data changes you
don't have to worry about integrating correct values into the text.

An example of this is the mean of the IQ data we have is 98.3387097


\end{document}
